\documentclass{article}

\input{padrao}

\title{Revisão Literária}
\author{Ivan de Souza Rehder}
\date{April 2023}

\begin{document}

\maketitle

% https://www.griffith.edu.au/griffith-sciences/school-environment-science/research/systematic-quantitative-literature-review

%\section{Being Systematic}
\section{Sendo sistemático}

%\subsection{Defining a topic}
\subsection{Definindo o tópico}

Uso de inteligência artificial para assistência ao piloto em cenário de loyal wingman

%\subsection{Research Questions}
\subsection{Perguntas de pesquisa}
\begin{itemize}
    \item \textit{Quando e quem já pesquisou o tema?;}
    \item \textit{Distribuição geográfica da pesquisa;}
    \item \textit{Quais métodos utilizados?;}
    \item \textit{Quem foram os voluntários utilizados?;}
    \item \textit{Quais variáveis foram estudadas?;}
    \item \textit{Quais foram as disciplinas que abordaram o tema?;}
    \item \textit{Há padrões encontrados nos resultados?} 
\end{itemize}

%\subsection{Key Words}
\subsection{Palavras-chave}

Sugestão da AEL: 
\begin{itemize}
    \item AI
    \item Cognitive Systems Engineering
    \item Loyal Wingman
    \item MUM-T: Manned Unmanned Team    
\end{itemize}  


\subsubsection{Bibliometrix}

%\subsection{Search Databases}
\subsection{Pesquisar em banco de dados}

SCOPUS
(TITLE-ABS-KEY(mum-t: OR (manned OR unmanned) AND "team*") OR TITLE-ABS-KEY ("loyal wingman")) AND (TITLE-ABS-KEY("cognitive systems") OR TITLE-ABS-KEY ("artificial intelligence" OR AI)) AND ( LIMIT-TO ( DOCTYPE,"cp" ) OR LIMIT-TO ( DOCTYPE,"ar" ) ) AND ( LIMIT-TO ( LANGUAGE,"English" ) )

212 resultados

%\subsection{Read and assess papers}
\subsection{Ler e avaliar os artigos}

\subsubsection{Ler e avaliar os artigos}

\citeonline{levulis2018effects} - Estuda o uso de interfaces multimodais (voz, toque) em uma operação MUM-T em 18 individuos controlando 3 UAV e monitorando outros 2 helicópteros tripulados. A conclusão é que o fato de ser multimodal não é vantajoso. Em média, a desempenho com apenas toque foi maior que a multimodal que foi maior que o com comando de voz. Possui algumas referências que abordam o tema de interfaces multimodais e algumas diretrizes de como escolher a interface para um problema.

%\section{Creating my own Database}
\section{Criando meu próprio banco de dados}

%\subsection{Structure Database}
\subsection{Estruturando o banco de dados}

%\subsection{Enter the first 10\% literatura}
\subsection{Os primeiros 10\% da literature}

%\subsection{Test and revise categories}
\subsection{Testando e revisando as categorias}

%\subsection{Enter bulk of papers}
\subsection{Um punhado de artigos}

%\subsection{Produce and review summary tables}
\subsection{Produzir e analisar as tabelas semânticas}

%\section{Writing the review}
\section{Escrevendo a análise}

%\subsection{Draft the methods}
\subsection{Rascunhar os métodos}

%\subsection{Evaluate and write the results}
\subsection{Avaliar e escrever os resultados}

%\subsection{Draft the introduction}
\subsection{Rascunhar a introdução}

%\subsection{Draft the Discussion}
\subsection{Rascunhar a discussão}

%\subsection{Draft the Abstract}
\subsection{Rascunhar o abstract}
%
%\subsection{Revise the papper}
\subsection{Revisar o artigos}

\bibliographystyle{abntex2-alf}
\bibliography{bibliography}

\end{document}

