\documentclass{article}

% ------------------------------------------------------------------
% Pacotes
% ------------------------------------------------------------------

% MARGENS
\usepackage[lmargin=3cm,tmargin=2.5cm,rmargin=3cm,bmargin=2.5cm]{geometry}

% ABNT
\usepackage[onehalfspacing]{setspace}
\usepackage[brazil]{babel}
% Essencial
\usepackage{xcolor,comment,enumerate,multirow,multicol}
% Criar lista com letras ou numéros
\usepackage{enumitem}
% Matemática
\usepackage{amsmath,amsfonts,dsfont,mathtools}
% ABNT
\usepackage{blindtext}

% ---
% Criar hiperlinks
% ---
\usepackage[hidelinks,colorlinks=false]{hyperref}

\usepackage{textcomp}
\usepackage{gensymb}


% ---
% Bibliografia ABNT
% ---
\usepackage[num,abnt-emphasize=bf,abnt-etal-cite=2,abnt-etal-list=0,abnt-etal-text=it]{abntex2cite}
\usepackage[brazilian,hyperpageref]{backref}

% ---
% Adicionar páginas de um pdf ao texto
% ---
\usepackage{pdfpages}


% ---
% Pacotes fundamentais 
% ---
\usepackage[utf8]{inputenc}                                                     % Codificacao do documento (conversão automática dos acentos)
\usepackage[T1]{fontenc}                                                        % Selecao de codigos de fonte.

\usepackage{lmodern}                                                            % Usa a fonte Latin Modern
\usepackage{indentfirst}		                                                % Indenta o primeiro parágrafo de cada seção.
\usepackage{nomencl} 			                                                % Lista de símbolos
\usepackage{color}				                                                % Controle das cores
\usepackage{graphicx, placeins}                                                 % Inclusão de gráficos
\usepackage{tabularx}       
\usepackage{booktabs}
\usepackage{adjustbox}
% Inclusão de tábelas
\usepackage{subfigure}
\usepackage{wrapfig}
\usepackage{microtype} 			                                                % para melhorias de justificação

% ---
% Fazer paragrafos com códigos
% ---
\usepackage{listings}

% ---
% Criação de um indice
% ---
\usepackage{imakeidx}

% ---
% Formatação dos capitulos e seções
% ---
\usepackage{titlesec}
    \titleformat{\section} %command
        [hang] %shape
        {\bfseries\Large} %format
        {\textbf{\thesection.\hspace{1em}}} %label
        {0.5ex} %sep
        {\vspace{0ex}} %before code
        [\vspace{0ex}] %% after code

    \titleformat{\subsection} %command
        [hang] %shape
        {\bfseries\large} %format
        {\textbf{\thesubsection \hspace{0.5em} - \hspace{0.5em}}} %label
        {0.5ex} %sep
        {\vspace{0ex}} %before code
        [\vspace{0ex}] %% after code


    %\titleformat{\subsubsection} %command
    
    %    [hang] %shape
    
    %    {\bfseries\medium} %format
    
    %    {\textbf{\hspace{1em}}} %label
    
    %    {0.5ex} %sep
    
    %    {\vspace{0ex}} %before code
    
    %    [\vspace{0ex}] %% after code
    
    
    %\titlespacing{\section}{12pc}{1.5ex plus .1ex minus .2ex}{1pc}

% ------------------------------------------------------------------
% Comandos
% ------------------------------------------------------------------
    
% Keywords command
\providecommand{\keywords}[1]
{
  \small	
  \textbf{\textit{Palavras-chave:}} #1
}



\title{Revisão Literária}
\author{Ivan de Souza Rehder}
\date{April 2023}

\begin{document}

\maketitle

% https://www.griffith.edu.au/griffith-sciences/school-environment-science/research/systematic-quantitative-literature-review

%\section{Being Systematic}
\section{Sendo sistemático}

%\subsection{Defining a topic}
\subsection{Definindo o tópico}

Uso de inteligência artificial para assistência ao piloto em cenário de loyal wingman

%\subsection{Research Questions}
\subsection{Perguntas de pesquisa}
\begin{itemize}
    \item \textit{Quando e quem já pesquisou o tema?;}
    \item \textit{Distribuição geográfica da pesquisa;}
    \item \textit{Quais métodos utilizados?;}
    \item \textit{Quem foram os voluntários utilizados?;}
    \item \textit{Quais variáveis foram estudadas?;}
    \item \textit{Quais foram as disciplinas que abordaram o tema?;}
    \item \textit{Há padrões encontrados nos resultados?} 
\end{itemize}

%\subsection{Key Words}
\subsection{Palavras-chave}

Sugestão da AEL: 
\begin{itemize}
    \item AI
    \item Cognitive Systems Engineering
    \item Loyal Wingman
    \item MUM-T: Manned Unmanned Team    
\end{itemize}  


\subsubsection{Bibliometrix}

%\subsection{Search Databases}
\subsection{Pesquisar em banco de dados}

SCOPUS
(TITLE-ABS-KEY(mum-t: OR (manned OR unmanned) AND "team*") OR TITLE-ABS-KEY ("loyal wingman")) AND (TITLE-ABS-KEY("cognitive systems") OR TITLE-ABS-KEY ("artificial intelligence" OR AI)) AND ( LIMIT-TO ( DOCTYPE,"cp" ) OR LIMIT-TO ( DOCTYPE,"ar" ) ) AND ( LIMIT-TO ( LANGUAGE,"English" ) )

212 resultados

%\subsection{Read and assess papers}
\subsection{Ler e avaliar os artigos}

\subsubsection{Ler e avaliar os artigos}

\citeonline{levulis2018effects} - Estuda o uso de interfaces multimodais (voz, toque) em uma operação MUM-T em 18 individuos controlando 3 UAV e monitorando outros 2 helicópteros tripulados. A conclusão é que o fato de ser multimodal não é vantajoso. Em média, a desempenho com apenas toque foi maior que a multimodal que foi maior que o com comando de voz. Possui algumas referências que abordam o tema de interfaces multimodais e algumas diretrizes de como escolher a interface para um problema.

%\section{Creating my own Database}
\section{Criando meu próprio banco de dados}

%\subsection{Structure Database}
\subsection{Estruturando o banco de dados}

%\subsection{Enter the first 10\% literatura}
\subsection{Os primeiros 10\% da literature}

%\subsection{Test and revise categories}
\subsection{Testando e revisando as categorias}

%\subsection{Enter bulk of papers}
\subsection{Um punhado de artigos}

%\subsection{Produce and review summary tables}
\subsection{Produzir e analisar as tabelas semânticas}

%\section{Writing the review}
\section{Escrevendo a análise}

%\subsection{Draft the methods}
\subsection{Rascunhar os métodos}

%\subsection{Evaluate and write the results}
\subsection{Avaliar e escrever os resultados}

%\subsection{Draft the introduction}
\subsection{Rascunhar a introdução}

%\subsection{Draft the Discussion}
\subsection{Rascunhar a discussão}

%\subsection{Draft the Abstract}
\subsection{Rascunhar o abstract}
%
%\subsection{Revise the papper}
\subsection{Revisar o artigos}

\bibliographystyle{abntex2-alf}
\bibliography{bibliography}

\end{document}

