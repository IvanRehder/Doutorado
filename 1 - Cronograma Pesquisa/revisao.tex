Uma revisão literária será feita para aumentar a familiaridade ao tema pesquisa. É esperado que ao final dessa etapa seja possível elaborar perguntas com relação a como ao campo estudado.

\subsection{Definir prompt de pesquisa nos bancos de dados}
\label{subsec:revisao_prompt}

Para pesquisar artigos em banco de dados os termos utilizados fazem muita diferença. Um termo muito abrangente pode ser pouco eficiente e um termo muito restrito pode cortar artigos significantes. Nessa etapa as definições do quais palavras-chaves são de interesse da pesquisa, em que momento elas foram publicas e a forma de publicação são definidas. 

\subsection{Leitura de abstracts}
\label{subsec:revisao_abstracts}

Momento de validação do pacote de trabalho anterior. É esperado que a tarefa anterior resulte em número muito alto de artigos para serem lidos na integra em curto tempo, mas um número considerável para ser lido apenas os abstracts. Essa pacote de trabalho então é uma leitura e classificação de todos os abstracts encontrados no pacote anterior baseado na relevância ou aproximação da pesquisa em questão. A maneira como essa relevância é definida é se o abstract apresentar sinal de adequação a temas relevantes para a pesquisa.

\subsection{Leitura completa de artigos}
\label{subsec:revisao_completa}

Agora com um número menor de artigos, a leitura completa desses artigos selecionados deve ser feita. É esperado que esses artigos selecionados sejam altamente relacionados a pesquisa, sendo importante para compreender o tema e em que momento o campo estudado se encontra atualmente.

\subsection{Refazer pesquisa e leitura de abstracts}
\label{subsec:revisao_refazer}

Esse é um pacote não obrigatório. Ele pode ser necessário se os artigos lidos no pacote de trabalho anterior podem indicarem que algumas palavras-chaves relevantes não foram utilizadas no pacote de trabalho \ref{subsec:revisao_prompt} ou a maneira de definir a relevância do \ref{subsec:revisao_abstracts} pode ser alterada. Se for o caso, os passos anteriores serão refeitos.

\subsection{Compilar resultados}
\label{subsec:revisao_compilar}

Após os todos os artigos relevantes serem lidos na integra, deve-se compilar todas as informações relevantes sobre o tema.

\subsection{Definir perguntas iniciais}
\label{subsec:revisao_perguntas}

Com as informações dos resultados, é possível nesse instante elaborar as primeiras perguntas sobre o tema para trilhar o rumo da pesquisa.

\subsection{Escrever relatório e apresentação}
\label{subsec:revisao_relatorio}

Esse pacote é sobre escrever um relatório e preparar uma apresentação contemplando todas as informações dos pacotes do entregável \ref{sec:revisao} explicando como foi realizado e quais foram os resultados colidos de cada pacote.