% ------------------------------------------------------------------
% Pacotes
% ------------------------------------------------------------------

% MARGENS
\usepackage[lmargin=3cm,tmargin=2.5cm,rmargin=3cm,bmargin=2.5cm]{geometry}

% ABNT
\usepackage[onehalfspacing]{setspace}
\usepackage[brazil]{babel}
% Essencial
\usepackage{xcolor,comment,enumerate,multirow,multicol}
% Criar lista com letras ou numéros
\usepackage{enumitem}
% Matemática
\usepackage{amsmath,amsfonts,dsfont,mathtools}
% ABNT
\usepackage{blindtext}

% ---
% Criar hiperlinks
% ---
\usepackage[hidelinks,colorlinks=false]{hyperref}

\usepackage{textcomp}
\usepackage{gensymb}


% ---
% Bibliografia ABNT
% ---
\usepackage[num,abnt-emphasize=bf,abnt-etal-cite=2,abnt-etal-list=0,abnt-etal-text=it]{abntex2cite}
\usepackage[brazilian,hyperpageref]{backref}

% ---
% Adicionar páginas de um pdf ao texto
% ---
\usepackage{pdfpages}


% ---
% Pacotes fundamentais 
% ---
\usepackage[utf8]{inputenc}                                                     % Codificacao do documento (conversão automática dos acentos)
\usepackage[T1]{fontenc}                                                        % Selecao de codigos de fonte.

\usepackage{lmodern}                                                            % Usa a fonte Latin Modern
\usepackage{indentfirst}		                                                % Indenta o primeiro parágrafo de cada seção.
\usepackage{nomencl} 			                                                % Lista de símbolos
\usepackage{color}				                                                % Controle das cores
\usepackage{graphicx, placeins}                                                 % Inclusão de gráficos
\usepackage{tabularx}       
\usepackage{booktabs}
\usepackage{adjustbox}
% Inclusão de tábelas
\usepackage{subfigure}
\usepackage{wrapfig}
\usepackage{microtype} 			                                                % para melhorias de justificação

% ---
% Fazer paragrafos com códigos
% ---
\usepackage{listings}

% ---
% Criação de um indice
% ---
\usepackage{imakeidx}

% ---
% Formatação dos capitulos e seções
% ---
\usepackage{titlesec}
    \titleformat{\section} %command
        [hang] %shape
        {\bfseries\Large} %format
        {\textbf{\thesection.\hspace{1em}}} %label
        {0.5ex} %sep
        {\vspace{0ex}} %before code
        [\vspace{0ex}] %% after code

    \titleformat{\subsection} %command
        [hang] %shape
        {\bfseries\large} %format
        {\textbf{\thesubsection \hspace{0.5em} - \hspace{0.5em}}} %label
        {0.5ex} %sep
        {\vspace{0ex}} %before code
        [\vspace{0ex}] %% after code


    %\titleformat{\subsubsection} %command
    
    %    [hang] %shape
    
    %    {\bfseries\medium} %format
    
    %    {\textbf{\hspace{1em}}} %label
    
    %    {0.5ex} %sep
    
    %    {\vspace{0ex}} %before code
    
    %    [\vspace{0ex}] %% after code
    
    
    %\titlespacing{\section}{12pc}{1.5ex plus .1ex minus .2ex}{1pc}

% ------------------------------------------------------------------
% Comandos
% ------------------------------------------------------------------
    
% Keywords command
\providecommand{\keywords}[1]
{
  \small	
  \textbf{\textit{Palavras-chave:}} #1
}

